%%%%%%%%%%%%%%%%%%%%%%%%%%%%%%%%%%%%%%%
% One Page Two Column Resume
% LaTeX Template
% Version 1.2 (16/9/2014)
%
% Author:
% Noah Huesser (yatekii@yatekii.ch)
% Version 1.1 (2016-04-03)
%
% IMPORTANT: THIS TEMPLATE NEEDS TO BE COMPILED WITH XeLaTeX
%
%%%%%%%%%%%%%%%%%%%%%%%%%%%%%%%%%%%%%%

\documentclass[]{resume}
\usepackage{polyglossia}
\setmainlanguage{english}
\usepackage[export]{adjustbox}
\begin{document}

%%%%%%%%%%%%%%%%%%%%%%%%%%%%%%%%%%%%%%
% P R O F I L E
%%%%%%%%%%%%%%%%%%%%%%%%%%%%%%%%%%%%%%

\begin{facts}
\section{Profile Of}
\section{Noah Hüsser}
%\LARGE\textit{Noah Hüsser}\normalsize
\sectionsep
%\includegraphics[valign=t, width=0.5\textwidth,left]{images/portrait_noah}

\subsection{Basics}
Nationality: CH\\
Date of Birth: 12th January 1991
\sectionsep

\subsection{Contact}
yatekii@yatekii.ch\\
+41 79 960 7130\\
\sectionsep
Eggenstrasse 3\\
5616 Meisterschwanden\\
Switzerland
\sectionsep

\subsection{Activities}
Ju Jitsu, Programming,\\
Electronics, Reading
\sectionsep

%%%%%%%%%%%%%%%%%%%%%%%%%%%%%%%%%%%%%%
% S K I L L S
%%%%%%%%%%%%%%%%%%%%%%%%%%%%%%%%%%%%%%

\section{Skills}

\subsection{Professional}
Software Development\\
Embedded Systems\\
FPGA Programming\\
Electronical Prototyping\\
Problem Solving\\
Project Management
\sectionsep

\subsection{Languages}
German \describe{mother tongue}\\
English \describe{fluent}\\
French  \describe{experienced}
\sectionsep

\section{Tools}

\subsection{Engineering}
Altium/KiCad\\
Shell\\
Jupyter, Scipy, Numpy, Pandas\\
MATLAB\\
Inventor
\sectionsep

\subsection{Programming}
\describe{frequently used}\\
Python, C,\\
VHDL, TCL,\\
JavaScript, SQL, LaTeX
\sectionsep

\describe{used in the past}\\
Java, C\#, PHP, C++, Bash
\sectionsep

\subsection{Miscellaneous}
Git/SVN\\
Microsoft Office\\
Photoshop / InDesign
\sectionsep

\end{facts}
\begin{timeline}

%%%%%%%%%%%%%%%%%%%%%%%%%%%%%%%%%%%%%%
%     EDUCATION
%%%%%%%%%%%%%%%%%%%%%%%%%%%%%%%%%%%%%%

\section{Education}

\subsection{FHNW Brugg-Windisch}
\descript{BsC in EE and IT}
\location{February 2016 – present | Brugg-Windisch, CH}
\sectionsep

\subsection{ETH Zürich}
\descript{BsC in Electrical Engineering and Information Technology}
\location{September 2013 – February 2016 | Zürich, CH}
\sectionsep

\subsection{Alte Kantonsschule Aarau}
\descript{Matura}
\location{Graduated 2012 | Aarau, CH}
\sectionsep

%%%%%%%%%%%%%%%%%%%%%%%%%%%%%%%%%%%%%%
%     EXPERIENCE
%%%%%%%%%%%%%%%%%%%%%%%%%%%%%%%%%%%%%%

\section{Practical Experience}

\subsection{ABB Micafil}
\descript{Intern Software Development }
\location{July 2016 – present | Altstetten, CH}
Development of various APIs and a CAD tool in VB/C\#.
\sectionsep

\subsection{Bastli}
\descript{President }
\location{February 2016 – October 2016 | Zürich, CH}
\descript{Quaestor }
\location{February 2015 – October 2016 | Zürich, CH}
\descript{Active Member }
\location{September 2013 – present | Zürich, CH}
Bastli is the student's electronics lab at ETH Zürich.
With Bastli we realize various, cool engineering projects. Most of the things we create are assembled from parts salvaged from the Junkyard.
Impressions on what we do at \textit{www.bastli.ch}.
\sectionsep

\subsection{Nexus Telecom}
\descript{Low Level C Programmer }
\location{Mai 2014 – Mai 2015 | Zürich, CH}
Diverse work on their main C library with focus on porting it from 32 to 64 bit.
\sectionsep

\subsection{AFC AG - Air Flow Consulting}
\descript{Internship as a Programmer }
\location{December 2012 – April 2013 | Zürich, CH}
Creation of various tools in Python, C\# and VB, working as a one man team.
\sectionsep

\subsection{wapple.ch}
\descript{Webdesigner and Co-Founder }
\location{Janury 2007  – present | Zürich, CH}
Webdesign for individuals, associations and companies in cooperation with a friend.
\sectionsep

%%%%%%%%%%%%%%%%%%%%%%%%%%%%%%%%%%%%%%
%     PUBLICATIONS
%%%%%%%%%%%%%%%%%%%%%%%%%%%%%%%%%%%%%%

\section{Publications and Projects}

\subsection{Design and Implementation of an FPGA Based Data Logging and Fault Recording System for PWM Rectifiers}
\descript{Group Thesis at ETH Zürich }
\location{September 2015  – December 2015 | Zürich, CH}
\vspace{\topsep} % Hacky fix for awkward extra vertical space
\begin{tightemize}
\item Recursive trigger logic to detect special signal patterns (VHDL)
\item Kernel module for data reading and processing on an ARM Core A9 (C)
\item GUI to retrieve data over the network, filter and display it (C++, Python, Qt5)
\end{tightemize}
\sectionsep

\subsection{230V Switch}
\descript{Bluetooth Controlled Two Channel 230V Switch, Private Project}
\location{November 2015  – present | Zürich, CH}
Design of PCB and Software that feature dimming, master/slave, timer and bluetooth control of ordinary power outlets. Proof of concept works.
\sectionsep

\subsection{backdoor}
\descript{Modular NFC Auth System, Private Project }
\location{December 2014  – June 2015 | Zürich, CH}
A system to manage access on devices such as the door to our backyard or a beer dispenser.
\sectionsep

%%%%%%%%%%%%%%%%%%%%%%%%%%%%%%%%%%%%%%
%     PERSONAL ACHIEVEMENTS
%%%%%%%%%%%%%%%%%%%%%%%%%%%%%%%%%%%%%%
\end{timeline}
\end{document}
